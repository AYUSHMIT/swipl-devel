\documentstyle[]{article}
\setlength{\textwidth}{6in}
\setlength{\textheight}{9in}
\setlength{\oddsidemargin}{1cm}
\setlength{\evensidemargin}{1cm}
\setlength{\parindent}{0in}
\setlength{\parskip}{5pt}
\nofiles
\begin{document}

\section*{XPCE/SWI-Prolog for Windows}

XPCE/SWI-Prolog for Windows is distributed in executable format for
non-commercial usage.  A licence allows for running an unlimited number
of copies on any machine owned by or in use by the licence holder.
Employees of the licence holder may run XPCE/SWI-Prolog on their
private machines.  Students of non-commercial academic institutions
may use XPCE/SWI-Prolog on their private machine if the institution
is a licence holder.

XPCE/SWI-Prolog requires an IBM compatible PC with a 386 or better
processor (486, Pentium) running MS-Windows 3.1.  The current system
requires 6 Megabytes for harddisk space, at least 8 Megabytes main
memory and a 3-button mouse.

XPCE/SWI-Prolog may be ordered by filling out this form and returning it
accompagnied by a cheque payable to SWI, University of Amsterdam, The
Netherlands.  The licence form and cheque should be sent to:

\begin{quote}
{\obeylines\parskip 0pt
 Ms. Zandvliet
 SWI
 University of Amsterdam
 Roetersstraat 15
 1018 WB~~Amsterdam
 The Netherlands.
}
\end{quote}

On receiving this licence form and a cheque, we will send you technical
details for downloading XPCE/SWI-Prolog if you requested distribution by
ftp or send you the system on floppy disk if you requested such.

The licence fee is dfl.  500,-- if the system is transferred by the
licence holder using ftp and dfl.  600,-- if distribution by floppy
disk is desired.  If you plan to transfer the software using ftp, please
verify you can contact our server: swi.psy.uva.nl (145.18.114.17).

Reference documentation is provided by a hyper-text system distributed
with XPCE/SWI-Prolog.  A user guide and a printable version of the
reference manual may be obtained for free using anonymous ftp to
the server named above in the directory pub/xpce/doc/...

For further information, please E-mail to xpce-request@swi.psy.uva.nl.

\newlength{\tag}
\settowidth{\tag}{Authorised Signature: }
\newlength{\rest}
\setlength{\rest}{\textwidth}
\addtolength{\rest}{-\tag}

\newcommand{\fillin}{\dotfill\mbox{}}
\newcommand{\onlydots}{\mbox{}\fillin}
\newcommand{\next}{\\[2mm]}

\vspace{0.5cm}
\makebox[\tag][l]{Licensee:}\fillin \next
\parbox[t]{\tag}{Name and address of \\ Institution:}%
\parbox[t]{\rest}{\onlydots \next \onlydots \next \onlydots} \next
\makebox[\tag][l]{VAT number:}\fillin \next
\makebox[\tag][l]{E-mail:}\fillin \next
\makebox[\tag][l]{Medium:}O 3.5'' MS-DOS formatted floppy disks (please include cheque for dfl. 600) \\
\makebox[\tag][l]{\mbox{}}O ftp (please include cheque for dfl. 500) \next
\makebox[\tag][l]{Date:}\fillin \next
\makebox[\tag][l]{Name:}\fillin \next
\makebox[\tag][l]{Authorised Signature:}\fillin \\


\section*{Software Release Agreement}

The undersigned, representing the institution identified below and
hereafter referred to as the Licensee, accepts the software known under
the name

\vspace{2mm}
\centerline{\bf XPCE/SWI-Prolog for Windows}
\vspace{2mm}

hereafter called the work, and agrees to the following conditions set
out below regarding its use and/or distribution.  The department of
Social Science Informatics (SWI) of the University of Amsterdam,
hereafter referenced as the Licensor, grants to the Licensee a
non-exclusive and non-transferrable licence to use the work only for
internal educational, evaluation and research activities.

The Licensee shall not distribute the work or any part thereof to others
or sell products or services based on the work, including educational
and research services, without the express permission of the Licensor.

\subsection*{Conditions of the Software release Agreement}

\begin{enumerate}
    \item[\it Prerequisites]
The licensee is responsible for obtaining any further licence that may
be necessary to provide the computing environment required by the said
work, such as for MS-Windows.  Rights implied by this Software Release
Agreement shall never exceed the rights implied by such further
licence(s) nor shall any rights implied by such further licence(s)
exceed the rights implied by this Software Release Agreement.
    \item[\it Limitations on use]
This Software Release Agreement does not permit the Licensee to use the
work or any part thereof for any commercial purposes.
    \item[\it Non-disclosure]
The licensee shall take all precaution to maintain the confidentiality
of the work; these precautions shall be at least equivalent to those
employed by the receiving organisation to protect its own confidential
information. 
    \item[\it Non-exclusivity]
The Licensee recognises that the work is released on a non-exclusive
basis and the Licensor shall have the exclusive right to grant licences
to others or to make such other use of the work as it shall desire.
    \item[\it Credits]
All credits in the work, both in listings and/or documentation, whether
names of individuals or organisations, will be retained in place by the
Licensee.
    \item[\it Product warranty]
The work is released on an ``as is'' basis, and there is no warranty
expressed or implied as to the functioning, performance or effect on
hardware or other software. The licensee recognises that the Licensor
is not obliged to provide maintenance, consultation or revision of the
work.

In case the work does not function on the Licensees hardware, all fees
will be returned if the Licensees destroys all copies of the work and
requests for a refund in writing within 1 month from the date of
purchase.
    \item[\it Future releases]
This licence grands the Licensees the right to transfer new releases of
the work free of charge using the internet ftp protocol for the period
of 1 year starting at the date of purchase if such new releases become
available.  The Licensor is not obliged to produce any new versions.
\end{enumerate}

\end{document}
